\documentclass[11pt]{article}
\usepackage[left=3cm,right=3cm,top=3cm,bottom=3cm]{geometry}
\usepackage{amssymb} 
\usepackage{enumitem}
\usepackage[fleqn]{amsmath} 
\usepackage{array}
\usepackage{bussproofs}
\usepackage{textgreek}
\usepackage{url}


\let\OLDthebibliography\thebibliography
\renewcommand\thebibliography[1]{
  \OLDthebibliography{#1}
  \setlength{\parskip}{0pt}
  \setlength{\itemsep}{0pt}
}
%https://texfaq.org/FAQ-compactbib.html

\begin{document}

\title{New Directions in Cryptography: Summary}
\author{Oliver Vecchini} 
\date{22\textsuperscript{nd} March 2019}

\maketitle


\begin{abstract}

        The following is a summary of the contents, context, and consequences
        of a seminal paper in theoretical computer science, namely '\textit{New
        Directions in Cryptography}', written by B. W. Diffie and M. E. Hellman
        and published in 1976 \cite{diffie1976new}. It is submitted as part of
        the assessed coursework component of the COMP0007 \textit{Directed
        Reading} module at UCL.

\end{abstract}



\section{Background}

Prior to the late 1970s, the task of secure communication over insecure channels
between two parties would have been achieved by one of many crytographic schemes
(common choices included the Lucifer and DES ciphers \cite{delfs2007symmetric}).
Regardless of choice, any scheme would invariably follow the same base 
protocol of \textit{secret-key encryption}, whose security relied on a
secret key being known to the two parties and those two \textit{only}. During
this period, the only way of achieving this would be generation of the key by
one, and transmission of it to the other over a secure channel; the existence of
a secure channel would seem to render the system redundant. In practice, the key
could be transmitted over an impractical but secure channel, and used several
times for secret-key encryption over an insecure channel, although this re-use
undermines security (for instance with the \textit{one-time pad}, the only
cipher in common use capable of unbreakable theoretical security, but reliant
on the key never being re-used \cite{delfs2007symmetric}).
\par
While the terminology of \textit{public-key} cryptography was introduced by
Diffie and Hellman's paper, many of the concepts it describes had been derived
earlier by others. By 1975, R. C. Merkle had derived the notion of exchanging a
key between two parties securely over an insecure channel, with cryptanalysis
rendered ineffective not due to its theoretical impossibility but its
computational infeasability, as well as providing an instance of one such scheme
\cite{merkle1978secure}; however, due to publication delays, his work was
published \textit{after} Diffie and Hellman's (interestingly, the first
publication of Merkle's paper was edited by R. L.  Rivest, one of the inventors
of the RSA public-key cipher, whose publication paper itself cites Merkle's
\cite{rivest1978method}). While theoretically significant, the scheme described
in his paper (later coined \textit{Merkle's Puzzles}) was unsuitable for
practical use; cryptanalysis of the scheme, while an inconvenience for an
attacker, was definitely tractable, while attempts at increasing security
required increasing channel bandwidth to unusable levels. 
\par
Prior even to this, researchers at GCHQ had not only proven the existence of a
scheme matching the description of public-key encryption, but had also derived
the exact procedures of the schemes later known as Diffie-Hellman key exchange
and the RSA cryptosystem, all realised in the 1970s prior to the publication of
Diffie and Hellman's paper, but not declassified until 1997
\cite{ellis1987story}.


\section{Contents of the article}

The authors begin by expressing the increasing importance of advancements in
cryptography, given the expansion of contemporary computer networks. An outline
of cryptographic convention until then is presented, outlining the operation of
secret-key cryptography, and the problems of privacy (assuring a sender that
only a particular party \textit{receives} a message) and one-way authentication
(assuring a recipient that only a particular party could have \textit{sent} a
message). Preference is expressed towards \textit{block} ciphers over
\textit{stream} ciphers (operating over messages totally and piecemeal
respectively), as minor changes in plaintext effect major changes in ciphertext
(\textit{error propogation}); this has benefits for authentication, preventing
attackers from splicing previous valid ciphertext into their own 'valid'
ciphertext.
\par
The novel material concept of the paper is the notion of \textit{public-key
cryptography}, motivated by the fact that with contemporary networks growing so
quickly, it would soon become impractical to establish a secure channel for
every pair of communicating parties; it would be preferable to communicate
exclusively over normal (but insecure) channels using public information to
establish private communications. Two possible schemes satisfying this are
shown, namely \textit{public-key cryptosystems} (PKCs) and \textit{public-key
distribution systems} (PKDs) introduced under the context of message privacy
(although PKCs are shown to relate to authentication as well). PKCs are defined
formally as sets of pairs of algorithms \( E_k, D_k \) for variable (seed) key
\( k \), which each transform a message into another message such
that
\begin{itemize}[noitemsep, nolistsep]
    \item \( E_k \) is the inverse of \( D_k \),
    \item \( E_k, D_k \) are computationally feasible for any message,
    \item Deriving one of \( D_k, k \) from \( E_k \) is generally
        computationally infeasible, and
    \item Deriving both \( E_k, D_k \) from \( k \) is computationally feasible.
\end{itemize}
As a result, if a party generates \( E_k, D_k \) from some key \( k \), 
\( E_k \) can be freely published to some publicly-available trusted repository
over a secure channel (as a 'public key'), needing only to keep \( D_k \) secret
(as a 'private key'). To send a private message, a sender can use the public key
of the intended recipient to encrypt it, and send the ciphertext to the
recipient over a normal (insecure) channel; only the recipient can decrypt it
using their private key. This eliminates the requirement for a secure channel
between sender and recipient! An example is provided wherein \( E_k, D_k \) are
represented by multiplication of an \( n \)-bit message vector with square
matrices \( A \) and \( B \) respectively, where \( A, B \) are inverse of
eachother. Generating two such matrices from a random key \( k \) takes \(
O(n^2) \) steps, while deriving one from the other takes \( O(n^3) \) steps
(although the factor of difference between the two is not large enough to yield
practically viable security). Another example is presented relying on security
by obscurity of \( E_k \)'s implementation, however this custom is heavily
frowned upon in modern practice \cite{citkerckhoff}.
\par
PKDs cover a somewhat simpler problem, but which remove the requirement of
secure channels nonetheless. The task is for two parties to reach consensus over
a key to be used for secret-key encryption without a secure channel; Merkle had
described the concept before and even provided a solution, although insecure.
Diffie and Hellman here present a new solution, based on the discrete logarithm
problem, or specifically the difficulty of calculating the integer \( k \) such
that for some \( a, b \in \mathbb{F}_p \) and prime \( p \), \( a = b^k \). The
procedure for each party is:
\begin{enumerate}[noitemsep, nolistsep]
    \item Agree with the other party on a prime \( p \) and some 
        \( \alpha \in \mathbb{F}_p \) over the insecure channel.
    \item Generate a random integer \( X_i \in \mathbb{F}_p \), which is kept
        private to the party.
    \item Calculate \( Y_i = \alpha{}^{X_i} \) and sends it to the other over
        the insecure channel.
    \item Calculate the same key across both parties using 
        \( K = Y_j^{X_i} \) (\( = \alpha{}^{X_iX_j} \)) (\( i \neq j \)).
\end{enumerate}
To perform cryptanalysis an attacker needs to derive \( K \) from \( Y_i,Y_j \);
the best method for this involves taking a discrete logarithm (e.g.  \( K =
Y_i^{log_\alpha{}Y_j} \)). This takes much longer than discrete exponentiation;
if the latter takes \( n \) steps, the former takes \( O(e^{cn}) \) steps (for
constant \( c \neq 0 \)), implying that cryptanalytic effort grows exponentially
relative to normal use.
\par
Formally defining this concept, the authors discuss \textit{one-way functions},
defined by the property of \textit{preimage resistance} (that is, for most given
outputs \( y \), it is computationally infeasible to find a value \( x \) that
maps to \( y \)), while remaining computationally feasible to evaluate.  These
are useful for storage of authentication information, with an example given in
the context of login systems with storing password hashes instead of as
plaintext. For general eavesdropping, the authors note that with a PKC, if a
party \( X \) encrypts a message with their own private key and publishes the
result (a \textit{signature}), other parties can decrypt the ciphertext with 
\( X \)'s public key to get a valid message, showing that only \( X \) could
have sent it; this scheme produces viable one-way authentication. Other
solutions are presented using one-way functions, although all are shown to be
impractical.
\par
The authors now link several of the concepts presented thus far to eachother.
Firstly, it is noted that if a block cipher \( B \) is secure against attackers
attempting to derive the secret key from any pair of plaintext/ciphertext, then
a high-quality one-way function \( f \) can be constructed from it such that
the domain of \( f \) is the keys of \( B \), and \( f \) applied to some 
\( k \) returns the ciphertext \( C \) generated by encrypting some random
message \( P \) with \( k \) (trying to infer \( k \) from \( C \) is, by
security of \( B \), infeasible); properties of this link (and alternative
constructions) are discussed (in particular the utility of one-way functions in
reasoning about cryptosystems). For the next connection, the authors coin some
definitions with relation to PKCs. A \textit{trap-door (one-way) function}
matches the definition of a one-way function except that, for a given \( y \),
it is easy to find some \( x \) that maps to \( y \) if certain
\textit{trap-door information} is known. It is thus clear to consider PKCs as
(bijective) trap-door functions whose evaluation, and derivation of inverse
with, and without trap-door information, reflect application of the public key,
application of the private key, and cryptanalysis respectively (the private key
\textit{is} the trap-door information). This also illustrates how PKCs can never
attain perfect theoretical security, as the private key is always uniquely
derivable from the public key. PKCs can also be used as PKDs (by transmitting a
chosen key), although the converse is not true; this highlights an interesting
duality between these two schemes and their underlying primitives, in that PKCs
are a subset of PKDs, while trap-door functions are a subset of one-way
functions.
\par
Finally, the authors discuss how to apply complexity theory to formally classify
computational infeasibility. They note the attractive property of
\textit{NP-complete} problems of being at least as hard as the hardest of
NP-hard problems (and are thus intractable unless \( P = NP \)). However,
analysis of these primarily concerns worst-case situations, while cryptographic
design favours analysis of average-case situations (i.e. security in all
cases); NP-complete problems are not necessarily intractible in the average
case. The authors express that a potential method of deriving a good one-way
function (and thus, a PKC by further adaption into a trap-door function) would
be to adapt some NP-complete problem such that no algorithm exists to solve the
problem in polynomial time in the average case, and there exists some algorithm
to solve the inverse problem in polynomial time for all input. The one-way
function is then the solution of the inverse problem. An example is given using
the NP-complete knapsack problem, and adapted to improve average-case
intractability, with flaws acknowledged. 


\section{Legacy}
The influence of the paper in the field of theoretical computer science has been
immeasurable, forming the root of an entire discipline of cryptography.  The
subsequent search for suitable trap-door one-way functions with intractible
inverses for PKCs bore fruit with the RSA cryptosystem a year later, whose
security is derived from the difficulty of factorising the product of two prime
numbers \cite{rivest1978method}. Another family of cryptographic schemes,
elliptic curve cryptography ('ECC'), uses methods based on the example PKD given
in the paper (since coined \textit{Diffie-Hellman key exchange}, or 'DH'), with
the advantage of smaller key size than RSA for the same security
\cite{lopez2000overview}.  These three schemes underpin the bulk of modern
public-key cryptography; they form one half of the TLS/SSL security protocol
(the \textit{de facto} standard for Web encryption, for instance in securing
HTTPS). Also directly derived from DH and the discrete logarithm problem are the
ElGamal cryptosystem and the Digital Signature Algorithm ('DSA')
\cite{dorey2017internet}.
\par
One unmentioned problem that would seem to be critical to PKCs is the issue of
authenticating a public key claiming to correspond to some party. The two chief
solutions addressing this are the (centralised) \textit{public-key
infrastructure} model, relying on \textit{certificate authorities} to attest to
key validity (as used by TLS), and the (decentralised) \textit{web of trust}
model, where authentic keys are acquired through networks of trusted parties (as
used by PGP) \cite{dorey2017internet}.
\par
As the paper recommends, several PKCs have been derived from NP-complete
problems, including the Merkle-Hellman knapsack, McEliece, and NTRUEncrypt
cryptosystems, with trap-door functions based on the knapsack, linear code
decoding, and shortest vector problem of lattices respectively.  While the first
of these has been broken, the latter two are distinct for being resistant to
attack by quantum computers, as all of the previously mentioned problems are
tractable using these with Shor's algorithm \cite{perlner2009quantum}.
Otherwise, the suggestion of deriving PKCs from NP-complete problems seems
unpopular, as many successful schemes use problems that are not even NP-hard
(such as factorisation in RSA, or the discrete logarithm problem of DH); many
now consider NP-completeness irrelevant to public-key cryptography
\cite{papadimitriou1997np}.
\par
Separate to the notions of PKCs and PKDs, the paper also stimulated development
of digital authentication schemes. Schemes developed since then include the
Rabin and Merkle signature schemes, as well as DSA; however, these all either
rely on trap-door functions or omit a proof of security. The first scheme not
reliant on trap-door functions that was proven to be secure was published by M.
Naor and M. M. Yung in 1989 \cite{naor1989universal}.

{\scriptsize
\bibliography{diffie1976new_report}
\bibliographystyle{unsrt}}

\end{document}

\documentclass[11pt]{article}
\usepackage[left=3cm,right=3cm,top=3cm,bottom=3cm]{geometry}
\usepackage{amssymb} 
\usepackage[fleqn]{amsmath} 
\usepackage{array}
\usepackage{bussproofs}
\usepackage{url}

%https://tex.stackexchange.com/a/214014
\newenvironment{bprooftree}
  {\leavevmode\hbox\bgroup}
  {\DisplayProof\egroup}

\linespread{0.6}

\let\OLDthebibliography\thebibliography
\renewcommand\thebibliography[1]{
  \OLDthebibliography{#1}
  \setlength{\parskip}{0pt}
  \setlength{\itemsep}{0pt}
}
%https://texfaq.org/FAQ-compactbib.html

\begin{document}

\setlength{\abovedisplayskip}{4pt}
\setlength{\belowdisplayskip}{7pt}
\setlength{\abovedisplayshortskip}{4pt}
\setlength{\belowdisplayshortskip}{7pt}

\title{An Axiomatic Basis for Computer Programming: \\Summary} 
\author{Oliver Vecchini} 
\date{31\textsuperscript{st} January 2019}

\maketitle


\begin{abstract}

        The following is a summary of the contents, context, and consequences of
        a seminal paper in theoretical computer science, namely '\textit{An
        Axiomatic Basis for Computer Programming}', written by C. A. R. Hoare in
        1969 \cite{hoare1969axiomatic}. It is submitted as a part of the
        assessed coursework component of the COMP0007 'Directed Reading' module
        at UCL.

\end{abstract}


\section{Background}

While it may seem apparent that the history of program correctness should go as
far back as the history of computer programming itself, even a short review of
history would appear to indicate otherwise. The first ever program written for
a stored-program computer, written by J. Von Neumann to illustrate usage of the
EDVAC (the first stored-program computer to be designed), itself contained
errors and did not work as intended: neither did the first non-trivial program
executed on the EDSAC (analogous to the  EDVAC) despite being written by the
lead of the EDSAC construction team M. V.  Wilkes.  Wilkes later came upon the
realization that 'a good part of the remainder of [his] life was going to be
spent in finding errors in [his] own programs' \cite{hayes1993discovery}. 
\par
The ease of writing error-ridden programs did not subside after this, however;
programming errors only increased in frequency and cost through to the 1960s,
with a notable example being the failure of the guidance systems of the
\textit{Mariner 1} spacecraft in 1962 due to a programming error, causing the
destructive abortion of its flight and over \$18.5 million in costs
(equivalent to over \$150 million in 2019) \cite{nytimesmariner}. The situation
was improving, with symbolic high-level programming languages mitigating errors;
however, the merits of structured programming were still hotly debated, with E.
W. Dijkstra's seminal letter naming the concept appearing only in 1968
\cite{dijkstra1968letters}; potentially dangerous programming constructs such as
jumps (explicitly noted by Hoare as difficult to reason about
\cite{hoare1969axiomatic}) remained in common use. 
\par 
It is interesting to examine the 'lineage' of papers addressing program
verification written before Hoare's own. Program verification was not a commonly
examined field of computer science, perhaps because of A. M. Turing's
demonstration that proving termination (and thus, total correctness) of a
general program was impossible \cite{turing1937computable}; however, J. McCarthy
set clear goals for the topic in 1961 (discussion reserved for next paper, where
content is more relevant). Several years later, two papers were written
independently by P. Naur \cite{naur1966proof} and R. W. Floyd
\cite{floyd1967assigning}, and published within a short period of eachother,
with striking similarity. Naur detailed an informal proof system of annotating
real programs (written in ALGOL 60) with informal statements about the
computer's state between each statement of the program, dubbed 'General
Snapshots' (akin to the \textit{inductive assertions} of the Hoare method);
however, no formal logical basis is given, and axioms/inference rules
are omitted from the 'proofs' (although triples of pre/postconditions and
programs are included, akin to those found in Floyd's and Hoare's papers). On
the other hand, Floyd details a system almost identical to that proposed in
Hoare's paper, complete with formal predicate logic and axioms/inference rules
(including the concept of the loop invariant), but tailored to flowcharts
instead of textual programs. Floyd's intention of the paper was more of an
incentive for development of machine-independent semantics (which would allow
proofs of this kind) rather than the development of an actual proof system. 
\par
All of the above are cited by Hoare as influences on his paper, but of
additional interest are papers written before even the earliest of these
reflecting the same sentiments. Von Neumann and H. H. Goldstine, in
writing the first 'textbook' of computation in the U.S.
\cite{goldstine1947planning}, describe several algorithms, each in terms of
flowcharts (indeed, the first ever application of these to programming); they
annotate the edges of these with 'assertions', as well as approximately
describing an inference rule for deriving these across edges. While serving as
more of a side-note than of the main body of the paper, it illustrates the
consideration of algorithm correctness well before the previously cited
material. Another paper with similar material was penned by Turing in 1949
\cite{turing1989checking}; however, the assertions, while fairly formal, were
neither expressed in a particular logic, nor had strict rules of inference to
connect them across program statements. 
\par

\section{Contents of the article}

In his paper \cite{hoare1969axiomatic}, Hoare builds upon the work of his peers
to provide a practical system for proving properties of programs: specifically,
one both relevant to text-based procedural programming languages
\cite{naur1966proof} and based in formal (here, predicate) logic
\cite{floyd1967assigning}. Hoare states that the goals of this are to not only
to provide such a system for its application alone, but also to provide impetus
for programming languages of the time to properly (i.e. axiomatically) define
the semantics of various parts of the language, at a time when many of these
were implementation-defined. 
\par
Hoare first defines a small set of axioms to describe Peano arithmetic on
non-negative integers in the context of computation (adapted from A. van
Wijngaarden \cite{van1966numerical}), generalised to abstract particulars of
implementation. Hoare then notes that the result of a program can be expressed
with assertions regarding the values of variables, after the program has
executed; to use this as a tool in proving properties of program, Hoare
describes a construct consisting of precondition, program, and postcondition 
(written in modern texts as \textit{Hoare triples}) expressed in the form $
\{P\} S \{Q\} $, where S is a program, and P and Q are statements formed of
predicate logic combined with the axioms of computer arithmetic, representing
the pre- and post-conditions respectively. These Hoare triples express the
notion that if P is true before executing S, then R will be true upon completion
of S. Hoare stresses the '\textbf{upon completion}' requirement, as developing a
system for proving total correctness would be fruitless (as corollary of the
halting problem), hence Hoare focuses on a general system for partial
correctness.
\par
Hoare then defines the axiom (schema) $D0$, expressed here (in modern syntax):
\[
    \begin{bprooftree}
        \LeftLabel{$D0$}
        \AxiomC{}
        \UnaryInfC{$ \vdash \{A[a/x]\} S \{A\} $}
    \end{bprooftree} 
\]

As well as the rules of inference $D1$, $D2$, and $D3$ (ditto): 

\[
    \noindent\begin{tabular}{l c r}
        \begin{bprooftree}
            \LeftLabel{$D1_a$}
            \AxiomC{$ \vdash P \implies P' $}
            \AxiomC{$ \vdash \{P'\} S \{Q\} $}
            \BinaryInfC{$ \vdash \{P\} S \{Q\} $}
        \end{bprooftree} 
        &
        \begin{bprooftree}
            \LeftLabel{$D1_b$}
            \AxiomC{$ \vdash Q \implies Q' $}
            \AxiomC{$ \vdash \{P\} S \{Q\} $}
            \BinaryInfC{$ \vdash \{P\} S \{Q'\} $}
        \end{bprooftree} 
        \\[1cm]
        \begin{bprooftree}
            \LeftLabel{$D2$}
            \AxiomC{$       \vdash \{P\} S_1 \{Q\} $}
            \AxiomC{$       \vdash \{Q\} S_2 \{R\} $}
            \BinaryInfC{$   \vdash \{P\} S_1; S_2 \{R\} $}
        \end{bprooftree} 
        &
        \begin{bprooftree}
            \LeftLabel{$D3$}
            \AxiomC{$ \vdash \{I \land C\} S \{I\} $}
            \UnaryInfC{$ \vdash \{I\} $ \textbf{while} $C$ \textbf{do} $S$ $ 
                       \{\lnot C \land I\} $}
        \end{bprooftree}
    \end{tabular}
\]

Where $D0$ is the \textit{Axiom of Assignment}, $D1$ is the \textit{Rule of
Consequence} (split into $D1_a$,  pre-condition strengthening, and $D1_b$,
post-condition weakening), $D2$ is the \textit{Rule of Composition}, and $D3$ is
the \textit{Rule of Iteration}; these are all taken from Floyd's paper
\cite{floyd1967assigning}, although $D0$ is 'reversed' in the sense that
substitution occurs here in the 'backwards' direction. Proofs of program
properties consist of a sequence of Hoare triples generated with the inference
rules, as the paper illustrates in an example.
\par
Hoare also discusses the benefits of automated program proof other than simply
producing correct programs. For instance, axiomatically specifying language
features gives stronger platform-independent guarantees about their behaviour,
while also explicitly indicating what behaviour \textit{is not} guaranteed,
making the task of language implementation more lenient; likewise, this concept
can be applied to the specifying and writing of particular programs/functions 
and \textit{their} expected behaviour. 

\section{Legacy}

As Hoare notes, his rules of inference can be supplanted with new rules
corresponding to new constructs, such as \textit{for} loops, \textit{it-else},
and \textit{skip}, which have all been added to the repertoire of modern Hoare 
logic. His own rules can also be 'strengthened', for instance as shown by
Dijkstra in strengthening the $D3$ rule \cite{dijkstra1978guarded}:

\[
    \noindent\begin{tabular}{l c r}
        \begin{bprooftree}
            \LeftLabel{$D3'$}
            \AxiomC{$ \vdash [I \land C \land t \in D \land t = z] S 
                    [I \land t \in D \land t < z]$}
            \UnaryInfC{$ \vdash [I \land t \in D] $ \textbf{while} $C$ 
                       \textbf{do} $S$ $ [\lnot C \land I \land t \in D] $}
        \end{bprooftree} 
    \end{tabular}
\]

This rule allows termination and hence total correctness to be proven in some
cases, using the principle of least integers and the well-founded ordering of
$<$ on $D$ (the change from $\{\}$ to $[]$ reflects the additional
guarantee of termination). Henceforth the term $t$ and predicate
$I$ of rule $D3$ are referred to as the \textit{loop variant} and
\textit{loop invariant}, respectively.
\par
It should be noted that the system Hoare proposes (\textit{Hoare logic}) cannot
automate the process of verifying either termination (corollary of the
undecidability of the halting problem) or partial verification (corollary of the
undecidability of non-trivial semantic properties of programs, i.e. Rice's
theorem \cite{rice1953classes}). However, Hoare logic \textit{does} allow for
human-assisted automated proofs, where the human assists with establishing loops
variants and invariants, which can be used to prove termination and partial
correctness, respectively. There is no guarantee that these are easy to find, or
even exist; for instance, the solution of whether the formula of the Collatz
conjecture terminates is dependent on finding a suitable loop variant (so far
unachieved \cite{andrei1998collatz}) and for partial correctness, Peano
arithmetic (and hence Hoare logic, which uses it) is incomplete, and hence there
exist true statements which cannot be proved.
\par
Hoare logic has seen great practical use in the time since its introduction. The
axiomatic system alone would continue to be used (by Hoare himself) in the
definition of the Pascal programming language \cite{hoare1973axiomatic}, but the
proof system would have the most influence, with numerous semi-automatic
implementations of Hoare logic created soon after, with early examples including
the Stanford Pascal verifier, Gypsy \cite{brocard2013programming}, and a
verifier written within the year of Hoare's paper (indeed, by a PhD student of
Floyd \cite{king1969program}). Separation logic, an extension of Hoare logic
to allow reasoning about shared mutable data manipulated by pointers
\cite{reynolds2002separation}, is also commonly used; modern examples of
verifiers using it include Smallfoot, SpaceInvader, Facebook's Infer (which in
particular verifies only particular properties about memory access/validity at
the benefit of full automation) \cite{fbresearch}, and Why3, a verification
'front-end' that uses SMT solvers to prove programs expressed in Hoare logic
\cite{filliatre2013why3}.  Prominent automated proofs using separation logic
include those of the $\mu$C/OS-II \cite{xu2016practical} and seL4
\cite{seL4_2019} operating system kernels.  Additionally, using the Curry-Howard
isomorphism between proofs and programs, it has been possible to
\textit{construct} programs \textit{from} proofs of desired properties,
expressed in Hoare logic \cite{poernomo2003curry}.
\par
Finally, the work of Hoare has spurred further research in verification in
domains other than Hoare logic; common proof assistants such as Isabelle/HOL and
Coq, the proof assistant behind the formally verified C compiler CompCert, all
operate on higher-order logic \cite{myreen2012proof}.

\section{In Review}

It would be unfair to attribute all of what is referred to as 'Hoare logic' to
this paper alone, given that it mostly just amalgamates the work of his peers as
noted above; however, given his explicit attribution of a portion of these peers
(and the possibility of unawareness of the others), added to the clarity of the
system he sets forth as well as the level of discussion of the benefits of such
a system, it is understandable that Hoare's paper would be the most
well-renowned of the collection.

\bibliography{hoare1969axiomatic_report} 
\bibliographystyle{unsrt}

\end{document}

\documentclass[11pt]{article}
\usepackage[left=3cm,right=3cm,top=3cm,bottom=3cm]{geometry}
\usepackage{amssymb} 
\usepackage[fleqn]{amsmath} 
\usepackage{array}
\usepackage{bussproofs}
\usepackage{textgreek}
\usepackage{url}


\let\OLDthebibliography\thebibliography
\renewcommand\thebibliography[1]{
  \OLDthebibliography{#1}
  \setlength{\parskip}{0pt}
  \setlength{\itemsep}{0pt}
}
%https://texfaq.org/FAQ-compactbib.html

\begin{document}

\title{Proving Properties of Programs by Structural Induction: \\Summary} 
\author{Oliver Vecchini} 
\date{21\textsuperscript{st} February 2019}

\maketitle


\begin{abstract}

        The following is a summary of the contents, context, and consequences of
        a seminal paper in theoretical computer science, namely '\textit{Proving
        Properties of Programs by Structural Induction}', written by R. M.
        Burstall and published in 1969 \cite{burstall1969proving}. It is
        submitted as a part of the assessed coursework component of the COMP0007
        'Directed Reading' module at UCL.

\end{abstract}


\section{Background}

As expressed in the previous paper in this series, program correctness has long
been a focus in both academic and industrial programming, with the costs of
program failure rising into the millions even by the 1960s (as in the case of
the destructive abortion of the \textit{Mariner 1} spacecraft in 1962). Formal
program verification has been of academic interest since the work of J. von
Neumann \& H. H.  Goldstine in 1947 and A. M. Turing in 1949, in
examining correctness of flowchart-based imperative programs; however, while
progress did continue for this paradigm (in particular under P. Naur and R. W.
Floyd), focus was also diverted towards correctness in the context of functional
programming \cite{olivecc}.
\par
The concepts found in functional programming languages are mostly derived from
those of \textlambda{}-calculus, introduced by A. Church in the 1930s
\cite{church1936unsolvable}, and combinatory logic, introduced by M.
Sch{\"o}nfinkel in 1924 \cite{schonfinkel1924bausteine}, two near-equivalent
notations which model computation in terms of the binding and evaluation of
functions. Focus was diverted from explicit mutability of data (e.g. assignment)
and control flow 'jumps', and towards modelling each program as the
evaluation of a single function. Programming languages were beginning to express
these concepts in the decade preceding Burstall's paper, with languages that
treat functions as first-class objects including IPL \cite{simon1996models},
ISWIM (used in Burstall's own paper \cite{burstall1969proving}), and LISP
\cite{mccarthy1959recursive}. The last of these was originally designed by J.
McCarthy, who continued on to make developments in the mathematical theory of
computation; one such contribution was \textit{recursion induction}, a system
for proving equivalence of two functions \(g\) and \(h\) by showing that, if
there exists some function \(f\) defined recursively by a particular equation
\(E\), and \(g\) and \(h\) both satisfy \(E\) by replacing \(f\), then \(g\) is
equivalent to \(h\) (this is illustrated in the domains of both generalised
computable functions and LISP).  McCarthy also defines \textit{S-expressions},
akin to inductively defined data structures, where each S-expression is either
an atom or an expression of S-expressions; McCarthy demonstrates application of
recursion induction to structures of this form as well \cite{mccarthy1963basis}. 
\par
While Burstall's paper was another step in the direction of program verification
in functional programming, it should be noted that the concepts expressed had
appeared in literature before. H.  B. Curry and R. Feys demonstrated the use of
structural induction for proofs concerning functions in combinatory logic in
1958 \cite{curry1958combinatory}, and McCarthy and J.  Painter used structural
induction (as an unnamed induction principle) to prove correctness of a compiler
in 1967 \cite{mccarthy1967correctness}.

\iffalse
difference to mccarthy/painter version?
read notes
\fi

\section{Contents of the article}

Burstall first discusses the variety of induction principles available for
program verification, and notes that structural induction is simply a
reformulated special case of recursion induction; he notes that reasoning about
programs with the former is generally easier than the latter (an opinion that
appears to be generally accepted \cite{brady1977hints}). Burstall briefly
describes the general 'roadmap' of the introduction to structural induction
(specifically aiming to be a middle-ground between an easily automatable
tool and an easily understood principle), as well as noting the programming
language used to illustrate proofs (ISWIM). 
\par
Burstall then describes the notion of inductively-defined data structures
(McCarthy's s-expressions are absent), where each \textit{object} of some type is
either an \textit{atom} or a \textit{structure}. An atom can be one of any 'base
objects', and a structure is an object returned by a \textit{constructor}
function evaluated with some number of objects as arguments; it is said that
these objects are the \textit{components} of the structure. Also defined are the
\textit{destructor} function, returning the components of a structure, and the
\textit{predicate} function, returning the truth value of whether a given object
was made with a given constructor. Note that these definitions match the
description of a general recursively-defined algebraic data type.
\par
Burstall then prepares to state the principle of structural induction. He
defines the \textit{constituency} relation,  where if an object is a constituent
of another object, then either the former is equal to the latter or the former
is a constituent of one of the components of the latter (i.e. the former is
involved in the construction of the latter); these objects also satisfy the
relation of \textit{proper constituency} if they are not equal. The principle of
structural induction is thus stated: for a set of objects, if, for all
objects, the fact that all proper constituents of an object satisfy a
property implies that the object itself satisfies the property, then all
objects in the set satisfy that property. This definition makes it
clear as to why proofs using structural induction often match the structure of
the functions they concern; similarly to how structures are 'built up' by
constructors with objects and ultimately atoms, the fact that a structure
satisfies a property is 'built up' from the fact that its proper constituents
(i.e. the objects and ultimately atoms used in constructing it) satisfy the
property.
\par
Of note is that Burstall arrives the principle of structural induction as a
specialization of Noetherian induction using the set of structures and the
relation of proper constituency. Only well-founded (i.e. all non-empty subsets
have a minimal element) partial ordering on a set is required for Noetherian
induction, as opposed to the well-ordering (and thus total ordering) required by
regular induction. This is required as proper constituency is \textit{not}
totally-ordered (for instance, two atoms are not proper constituents of one
another). 
\par
Burstall proceeds to introduce the ISWIM language, and adapt its syntax to make
written proofs even more transparent and clear; however, it should be noted that
this is only syntactic sugar and strictly is unrelated to the actual proof
mechanism of structural induction. He gives rules \textalpha{} and \textbeta{}
for variable name substitution to convert between equivalent program
expressions, for variable name changing and variable value forward-substitution
respectively. He groups several ISWIM operators into the single operator
\textit{cons}, which he changes to an infix \textit{::} operator; he further
advises this procedure for any general constructor operation. Burstall compares
McCarthy's axioms for proving LISP programs using recursion induction
\cite{mccarthy1963basis} with his own substitution rules, noting both their
generality across constructor functions (examples are given in terms of list
operations and the \textit{cons} operator, but are not specific to these) and
their flexibility for adding rules later. The pattern-matching idiom is proposed
here (although not named as such) to avoid redundancy in writing
constructor/destructor cases, again as syntactic sugar to make proof-writing
less tedious (although the formal rules for selecting the appropriate case are
omitted).  Also proposed is the notion of the constructor type signature,
consisting of the types of the objects taken as arguments by the constructor.
\par
To illustrate the use of structural induction and the adapted syntax of ISWIM,
Burstall proves statements regarding a \textit{fold}-style expression, a
tree-sorting function, and an expression-to-machine-code compiler function. In
each instance, he proves the property for each function argument \textit{case},
akin to the proving of the property for each of atoms and structures. The
compiler example in particular shows that expressions can be evaluated by
working with a stack, in particular that  executing a compiled expression is
equivalent to loading the value of that expression onto the stack. Using other
auxiliary functions defined earlier, this is proved completely using structural
induction.

\section{Legacy}

The influences of Burstall's paper are numerous. Firstly, the structural
induction principle has had considerable impact, with research continuing in its
automation conducted by R. Aubin \cite{aubin1979mechanizing}, Brotz
\cite{brotz1974embedding}, and R. S. Boyer \& J. S. Moore. The practical impact
of the paper is also evident; the work of Boyer \& Moore in particular lead to
the creation of the Nqthm/ACL2 automated theorem provers \cite{acl2}, and
examples of others include Zeno and Hipspec \cite{claessen2012hipspec}. Together
with McCarthy's work, Burstall's paper motivated the further general research of
functional program verification, with contributions by McCarthy, Z. Manna, and
A. Pneuli appearing in the 1970s \cite{manna1969properties}
\cite{manna1970formalization}, to name a few. Functional programming is now the
dominant paradigm of programming verification, with common tools such as Coq
and Isabelle serving as functional programming languages
\cite{yushkovskiy2018comparison}. Type theory has also been impacted, with the
notion of coinductive types stemming from structural induction
\cite{gimenez1998structural}. Finally, the augmented notation of Burstall's
paper has been widely adopted, with the function type signature convention, the
\textit{::} notation for \textit{cons}, and the pattern-matching idiom finding
their ways into popular languages \cite{ullman1994elements}.


\section{In Review}

Overall, the paper is extremely clear in its description of the concepts it
presents. Despite taking multiple concepts from earlier work
\cite{mccarthy1963basis} \cite{curry1958combinatory}, everything is presented
without requiring prior knowledge in the field, gradually and thoroughly enough
that one unacquainted with the topic can easily approach the paper. Despite a
few inconsistencies of terminology (using 'constituent' to refer to 'proper
constituents' such as in the phrase 'no structure is identical to a constituent
of itself', and mixing up of the phrases 'object' and 'structure' when strictly
a structure is just one possible form of an object), Burstall clearly (and
explicitly) recognizes the benefit of clarity in order to ensure future
development of the subject. He attributes the previous work of others clearly,
while also highlighting areas in which progress could continue. Burstall's paper
is an excellent groundwork for further research into both the specific topic of
structural induction and the general area of program verification.


\bibliography{burstall1969proving_report} 
\bibliographystyle{unsrt}

\end{document}
